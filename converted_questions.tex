\begin{question}
\type{MCQ_SINGLE}
\chapter{DEFAULT_CHAPTER}
\difficulty{MODERATE}
\text{This section contains FOUR (04) questions.}
\answer{?}
\end{question}

\begin{question}
\type{MCQ_SINGLE}
\chapter{DEFAULT_CHAPTER}
\difficulty{MODERATE}
\text{Each question has FOUR options (A), (B), (C) and (D). ONLY ONE of these four options is the correct answer.}
\answer{?}
\end{question}

\begin{question}
\type{MCQ_SINGLE}
\chapter{DEFAULT_CHAPTER}
\difficulty{MODERATE}
\text{For each question, choose the option corresponding to the correct answer.}
\answer{?}
\end{question}

\begin{question}
\type{MCQ_SINGLE}
\chapter{DEFAULT_CHAPTER}
\difficulty{MODERATE}
\text{Answer to each question will be evaluated according to the following marking scheme : \end{itemize} \begin{center} \begin{tabular}{llll} Full Marks & $:$ & +3 & If ONLY the correct option is chosen. \\ Zero Marks & $:$ & 0 & If none of the options is chosen (i.e. the question is unanswered) \\ Negative Marks & $:$ & -1 & In all other cases \\ \end{tabular} \end{center}}
\answer{?}
\end{question}

\begin{question}
\type{MCQ_SINGLE}
\chapter{DEFAULT_CHAPTER}
\difficulty{MODERATE}
\text{A thin lens of material having refractive index $\mu=1.5$ and focal length of 20 cm in air has two mediums of different refractive indices $\mu_{1}=1.2$ and $\mu_{2}=2.5$ cover upper and lower halves of the lens, respectively as shown in figure. If an object is placed on the principal axis, then its two images will form; one after refraction from upper part and other after refraction from lower part. Consider the object to be at $\infty$, the separation between two images formed would be :-\\ \includegraphics[max width=\textwidth, center]{34fecb1e-3dd5-40ba-8a82-f13b292ab7b9-03}\\}
\option{A}{15 cm\\}
\option{B}{40 cm\\}
\option{C}{25 cm\\}
\option{D}{65 cm}
\answer{?}
\end{question}

\begin{question}
\type{MCQ_SINGLE}
\chapter{DEFAULT_CHAPTER}
\difficulty{MODERATE}
\text{Consider a rod AB of length L whose mass per length is given by $\lambda=(1+\mathrm{Kx})$ where K is a constant and x is the distance from the left end A . If the moment of inertia of the rod will be least about an axis passing through point P , then AP is equal to :-\\}
\option{A}{$\left(\frac{3 \mathrm{~L}+2 \mathrm{KL}^{2}}{6+3 \mathrm{KL}}\right)$\\}
\option{B}{$\left(\frac{2 \mathrm{~L}+3 \mathrm{KL}^{2}}{6+2 \mathrm{KL}}\right)$\\}
\option{C}{$\left(\frac{3 \mathrm{~L}+\mathrm{KL}^{2}}{4+3 \mathrm{KL}}\right)$\\}
\option{D}{$\left(\frac{3 \mathrm{~L}+2 \mathrm{KL}^{2}}{4+3 \mathrm{KL}}\right)$}
\answer{?}
\end{question}

\begin{question}
\type{MCQ_SINGLE}
\chapter{DEFAULT_CHAPTER}
\difficulty{MODERATE}
\text{A bead is connected with a fixed disc of radius R by an inextensible massless string in a smooth horizontal plane. If the bead is pushed with a velocity $\mathrm{v}_{0}$ perpendicular to the string, the bead moves in a curve and consequently collapses on the disc. Then :\\ \includegraphics[max width=\textwidth, center]{34fecb1e-3dd5-40ba-8a82-f13b292ab7b9-04}\\}
\option{A}{For the particle $\frac{\mathrm{d}^{2} \theta}{\mathrm{dt}^{2}}=\frac{\mathrm{v}_{0}^{2}}{\ell^{2}}$ where $\theta$ is the angle by which particle turns.\\}
\option{B}{For the particle $\frac{\mathrm{d}^{2} \theta}{\mathrm{dt}^{2}}=\frac{\mathrm{v}_{0}^{2} \mathrm{R}}{\ell^{3}}$ where $\theta$ is the angle by which particle turns.\\}
\option{C}{Distance travelled by the particle till it collides with the disc is $\frac{\ell^{2}}{4 R}$\\}
\option{D}{Distance travelled by the particle till it collides with the disc is $\frac{\ell^{2}}{R}$}
\answer{?}
\end{question}

\begin{question}
\type{MCQ_SINGLE}
\chapter{DEFAULT_CHAPTER}
\difficulty{MODERATE}
\text{A hemisphere of radius R and refractive index $\sqrt{2}$ is placed in a liquid of refractive index $\sqrt{3}$ as shown in figure. A ray is incident at point P at an angle $60^{\circ}$. As the ray emerges out from the hemisphere the angle of refraction is $0^{\circ}$. The distance of the point P from A is\\ \includegraphics[max width=\textwidth, center]{34fecb1e-3dd5-40ba-8a82-f13b292ab7b9-04(1)}\\}
\option{A}{$\sqrt{2} \mathrm{R}$\\}
\option{B}{$\sqrt{3} \mathrm{R}$\\}
\option{C}{$\frac{2 \mathrm{R}}{\sqrt{3}}$\\}
\option{D}{2 R}
\answer{?}
\end{question}

\begin{question}
\type{MCQ_MULTIPLE}
\chapter{DEFAULT_CHAPTER}
\difficulty{MODERATE}
\text{This section contains THREE (03) questions.}
\answer{?}
\end{question}

\begin{question}
\type{MCQ_MULTIPLE}
\chapter{DEFAULT_CHAPTER}
\difficulty{MODERATE}
\text{Each question has FOUR options. ONE OR MORE THAN ONE of these four option(s) is (are) correct answer(s).}
\answer{?}
\end{question}

\begin{question}
\type{MCQ_MULTIPLE}
\chapter{DEFAULT_CHAPTER}
\difficulty{MODERATE}
\text{For each question, choose the option(s) corresponding to (all) the correct answer(s)}
\answer{?}
\end{question}

\begin{question}
\type{MCQ_MULTIPLE}
\chapter{DEFAULT_CHAPTER}
\difficulty{MODERATE}
\text{Answer to each question will be evaluated according to the following marking scheme: \end{itemize} \begin{center} \begin{tabular}{|l|l|l|} \hline Full Marks & : +4 & If only (all) the correct option(s) is (are) chosen. \\ \hline Partial Marks & : +3 & If all the four options are correct but ONLY three options are chosen. \\ \hline Partial Marks & : +2 & If three or more options are correct but ONLY two options are chosen and both of which are correct. \\ \hline Partial Marks & : +1 & If two or more options are correct but ONLY one option is chosen and it is a correct option. \\ \hline Zero Marks & : 0 & If none of the options is chosen (i.e. the question is unanswered). \\ \hline Negative Marks & : -2 & In all other cases. \\ \hline \end{tabular} \end{center} \begin{itemize}}
\answer{?}
\end{question}

\begin{question}
\type{MCQ_MULTIPLE}
\chapter{DEFAULT_CHAPTER}
\difficulty{MODERATE}
\text{For Example : If first, third and fourth are the ONLY three correct options for a question with second option being an incorrect option; selecting only all the three correct options will result in +4 marks. Selecting only two of the three correct options (e.g. the first and fourth options), without selecting any incorrect option (second option in this case), will result in +2 marks. Selecting only one of the three correct options (either first or third or fourth option), without selecting any incorrect option (second option in this case), will result in +1 marks. Selecting any incorrect option(s) (second option in this case), with or without selection of any correct option(s) will result in -2 marks. \end{itemize} \setcounter{enumi}{4}}
\answer{?}
\end{question}

\begin{question}
\type{MCQ_MULTIPLE}
\chapter{DEFAULT_CHAPTER}
\difficulty{MODERATE}
\text{A light ray is incident on equilateral prism of side length d on mid point at variable angle of incidence i. Now choose the correct answer(s). [ D is mid point of side AB]\\ \includegraphics[max width=\textwidth, center]{34fecb1e-3dd5-40ba-8a82-f13b292ab7b9-05}\\}
\option{A}{If $\mu=\sqrt{2}$, minimum value of distance AE is (without any TIR from side AC) $\frac{d}{2}(\sqrt{3}-1)$.\\}
\option{B}{If $\mu=\sqrt{2}$, angle of minimum deviation produced by prism is $30^{\circ}$.\\}
\option{C}{If $\mu<2$, then and only then, there may be emergent ray from AC of prism (without TIR)\\}
\option{D}{In case of minimum deviation emergent ray must be emerging from midpoint of face AC.}
\answer{?}
\end{question}

\begin{question}
\type{MCQ_MULTIPLE}
\chapter{DEFAULT_CHAPTER}
\difficulty{MODERATE}
\text{A spherical surface separates air \& medium for which $\mu=1.615$ for violet and $\mu=1.600$ for red color. A paraxial beam parallel to optic axis is incident on the surface as shown. The distance between point of convergence for violet and red color is $\Delta \mathrm{f}$.\\ \includegraphics[max width=\textwidth, center]{34fecb1e-3dd5-40ba-8a82-f13b292ab7b9-06}\\}
\option{A}{$\Delta \mathrm{f}=0.4 \mathrm{~cm}$ approx.\\}
\option{B}{Point of convergence for red is closer to optical centre than that for violet.\\}
\option{C}{Point of convergence for violet is closer to optical centre than that for red.\\}
\option{D}{$\Delta \mathrm{f}=0.84 \mathrm{~cm}$ approx.}
\answer{?}
\end{question}

\begin{question}
\type{MCQ_MULTIPLE}
\chapter{DEFAULT_CHAPTER}
\difficulty{MODERATE}
\text{The radius of the tire of a car is R . The valve cap is at a distance r from the axis of the wheel. The car starts from rest without skidding, at constant acceleration. It's found that the valve cap has no acceleration in the $\frac{1}{8}$ turn preceding the bottom most position. Select the correct alternative(s).\\}
\option{A}{The ratio $\frac{\mathrm{R}}{\mathrm{r}}=\sqrt{2}$\\}
\option{B}{The car has travelled a distance $\frac{R}{2}$ upto this time\\}
\option{C}{The wheel of the car has turned by 0.5 rad upto this time\\}
\option{D}{The car has travelled a distance $\frac{\mathrm{R}}{3}$ upto this time.}
\answer{?}
\end{question}

\begin{question}
\type{MCQ_SINGLE}
\chapter{DEFAULT_CHAPTER}
\difficulty{MODERATE}
\text{8. List-I gives different lens configurations. The radius of curvature of each surface is R. Rays of light parallel to the axis of lens from left of lens traversing through the lens get focused at distance ffrom the lens. List-II gives corresponding values of magnitudes of f ( $\mu$ represent refractive index) :- \\ \begin{center} \begin{tabular}{|l|l|l|l|} \hline & List-I & & List-II \\ \hline (P) \includegraphics[max width=\textwidth]{34fecb1e-3dd5-40ba-8a82-f13b292ab7b9-07} & & (1) 13 R \\ \hline (Q) Lens is silvered & & (2) $\frac{13 \mathrm{R}}{4}$ \\ \hline (R) Plano concave lens & & (3) $\frac{7 \mathrm{R}}{3}$ \\ \hline (S) \includegraphics[max width=\textwidth]{34fecb1e-3dd5-40ba-8a82-f13b292ab7b9-07(1)} & & (4) $\frac{7 \mathrm{R}}{16}$ \\ \hline  & & (5) None of these \\ \hline \end{tabular} \end{center}}
\option{A}{$\mathrm{P} \rightarrow 1 ; \mathrm{Q} \rightarrow 2 ; \mathrm{R} \rightarrow 3 ; \mathrm{S} \rightarrow 4$\\}
\option{B}{$\mathrm{P} \rightarrow 2 ; \mathrm{Q} \rightarrow 1 ; \mathrm{R} \rightarrow 3 ; \mathrm{S} \rightarrow 4$\\}
\option{C}{$\mathrm{P} \rightarrow 2 ; \mathrm{Q} \rightarrow 3 ; \mathrm{R} \rightarrow 4 ; \mathrm{S} \rightarrow 1$\\}
\option{D}{$\mathrm{P} \rightarrow 3 ; \mathrm{Q} \rightarrow 4 ; \mathrm{R} \rightarrow 1 ; \mathrm{S} \rightarrow 2$\\}
\answer{?}
\end{question}

\begin{question}
\type{MCQ_SINGLE}
\chapter{DEFAULT_CHAPTER}
\difficulty{MODERATE}
\text{9. In list-I, we have certain situations showing object and optical system. Match them with description of image in list-II.\\
\includegraphics[max width=\textwidth, center]{34fecb1e-3dd5-40ba-8a82-f13b292ab7b9-08}\\}
\option{A}{$\mathrm{P} \rightarrow 4 ; \mathrm{Q} \rightarrow 1 ; \mathrm{R} \rightarrow 3 ; \mathrm{S} \rightarrow 2$\\}
\option{B}{$\mathrm{P} \rightarrow 4 ; \mathrm{Q} \rightarrow 3 ; \mathrm{R} \rightarrow 4 ; \mathrm{S} \rightarrow 5$\\}
\option{C}{$\mathrm{P} \rightarrow 3 ; \mathrm{Q} \rightarrow 4 ; \mathrm{R} \rightarrow 5 ; \mathrm{S} \rightarrow 2$\\}
\option{D}{$\mathrm{P} \rightarrow 3 ; \mathrm{Q} \rightarrow 4 ; \mathrm{R} \rightarrow 5 ; \mathrm{S} \rightarrow 1$\\}
\answer{?}
\end{question}

\begin{question}
\type{MCQ_SINGLE}
\chapter{DEFAULT_CHAPTER}
\difficulty{MODERATE}
\text{10. In the picture structure S is shown. It may be a semi ring, semi disc, hollow hemisphere or solid hemisphere. Its mass is m and is placed on a sufficiently rough surface. It is released in the position shown in figure. In List-II values of few quantities are given which correspond to the situation written in List-I. Match the proper entries from List-I with List-II. (Use : $\pi^{2}=10, g=10 \mathrm{~m} / \mathrm{s}^{2}$ )\\
\includegraphics[max width=\textwidth, center]{34fecb1e-3dd5-40ba-8a82-f13b292ab7b9-09} \\ \begin{center} \begin{tabular}{|l|l|l|l|} \hline & List-I & & List-II \\ \hline (P) Homogeneous semicircular ring ( $\mathrm{m}=5 / 4 \mathrm{~kg}$ ) & & (1) Angular acceleration $=\mathrm{g} / \pi \mathrm{R}$ \\ \hline (Q) Homogenous semicircular disc ( $\mathrm{m}=135 / 119 \mathrm{~kg}$ ) & & (2) Magnitude of acceleration of centre of mass $=\frac{8 \mathrm{~g}}{9 \pi} \sqrt{1+\frac{16}{9 \pi^{2}}}$ \\ \hline (R) Homogeneous hollow hemisphere ( $\mathrm{m}=20 / 17 \mathrm{~kg}$ ) & & (3) Magnitude of normal contact force = 10 N \\ \hline (S) Homogeneous solid hemisphere ( $\mathrm{m}=448 / 403 \mathrm{~kg}$ ) & & (4) Minimum coefficient of friction = 6/17 \\ \hline  & & (5) Friction force acting $=2.98 \mathrm{~N}$ \\ \hline \end{tabular} \end{center}}
\option{A}{$\mathrm{P} \rightarrow 1,3 ; \mathrm{Q} \rightarrow 2,3 ; \mathrm{R} \rightarrow 3,4 ; \mathrm{S} \rightarrow 3,5$\\}
\option{B}{$\mathrm{P} \rightarrow 1,3 ; \mathrm{Q} \rightarrow 2,3,4 ; \mathrm{R} \rightarrow 1,2 ; \mathrm{S} \rightarrow 3,5$\\}
\option{C}{$\mathrm{P} \rightarrow 2,3 ; \mathrm{Q} \rightarrow 1,3 ; \mathrm{R} \rightarrow 3,4 ; \mathrm{S} \rightarrow 3,5$\\}
\option{D}{$\mathrm{P} \rightarrow 3,5 ; \mathrm{Q} \rightarrow 1,3 ; \mathrm{R} \rightarrow 2,3 ; \mathrm{S} \rightarrow 3,4$\\}
\answer{?}
\end{question}

\begin{question}
\type{INTEGER}
\chapter{DEFAULT_CHAPTER}
\difficulty{MODERATE}
\text{11. A roller of radius 5.0 cm rides between two horizontal bars moving with velocities $\mathrm{v}_{\mathrm{a}}$ and $\mathrm{v}_{\mathrm{b}}$ as shown in figure along the horizontal. Assume no slip conditions at the point A and B. Now match the entries given in List-I against the relevant entries given in List-II. (IAOR - instantaneous axis of rotation)\\ \includegraphics[max width=\textwidth, center]{34fecb1e-3dd5-40ba-8a82-f13b292ab7b9-10} \begin{center} \begin{tabular}{|l|l|l|l|} \hline \multicolumn{2}{|c|}{List-I} & \multicolumn{2}{|c|}{List-II} \\ \hline}
\answer{?}
\end{question}

\begin{question}
\type{INTEGER}
\chapter{DEFAULT_CHAPTER}
\difficulty{MODERATE}
\text{This section contains SIX (06) questions.}
\answer{?}
\end{question}

\begin{question}
\type{INTEGER}
\chapter{DEFAULT_CHAPTER}
\difficulty{MODERATE}
\text{The answer to each question is a NON-NEGATIVE INTEGER}
\answer{?}
\end{question}

\begin{question}
\type{INTEGER}
\chapter{DEFAULT_CHAPTER}
\difficulty{MODERATE}
\text{For each question, enter the correct integer value of the answer in the place designated to enter the answer.}
\answer{?}
\end{question}

\begin{question}
\type{INTEGER}
\chapter{DEFAULT_CHAPTER}
\difficulty{MODERATE}
\text{For each question, marks will be awarded in one of the following categories : \end{itemize} Full Marks : +4 If only the correct answer is given.\\ Zero Marks : 0 In all other cases.}
\answer{?}
\end{question}

\begin{question}
\type{INTEGER}
\chapter{DEFAULT_CHAPTER}
\difficulty{MODERATE}
\text{A 10 kg block B rests on smooth rollers of negligible mass and is in contact with a 2 kg solid cylinder of the radius 0.1 m as shown. The rollers ensure that there is no friction between B and the ground A below B . It is know that $\mu_{\mathrm{k}}=\mu_{\mathrm{s}}=0.5$ between B and C . Find the magnitude of the horizontal force P (in N ) for which the cylinder will roll with an angular acceleration of $20 \mathrm{rad} / \mathrm{s}^{2}$ on ground A without slipping. Fill $\frac{\mathrm{P}}{16}$.\\ \includegraphics[max width=\textwidth, center]{34fecb1e-3dd5-40ba-8a82-f13b292ab7b9-11}}
\answer{?}
\end{question}

\begin{question}
\type{INTEGER}
\chapter{DEFAULT_CHAPTER}
\difficulty{MODERATE}
\text{A concave mirror of radius of curvature 50 cm lies at origin of the coordinate system such that its axis coincides with x axis. A small straight object lies on the x -axis along the line $5 \mathrm{y}+4 \mathrm{x}-25=0$. Assuming mirror formula to be valid for the image formation of this object, calculate the value of 5 m where m is slope of the line along which the image lies.}
\answer{?}
\end{question}

\begin{question}
\type{INTEGER}
\chapter{DEFAULT_CHAPTER}
\difficulty{MODERATE}
\text{Two balls of equal density and radii r and $\mathrm{R}=2 \mathrm{r}$, are placed with the centre of the larger one directly above the middle of a cart of mass $M=6 \mathrm{~kg}$ and length $\mathrm{L}=2 \mathrm{~m}$. The mass of the smaller ball is $\mathrm{m}=1 \mathrm{~kg}$. The balls are made to roll, without slipping, in such a way that the larger ball rests on the cart and a straight line connecting their centres remains at a constant angle $\phi=60^{\circ}$ to the horizontal. The cart is pulled by a horizontal force in the direction shown in figure. If the time is $t$ which elapses before the balls fall off the cart then find $(t+0.45)$ in seconds. Round off to nearest integer.\\ \includegraphics[max width=\textwidth, center]{34fecb1e-3dd5-40ba-8a82-f13b292ab7b9-11(1)}}
\answer{?}
\end{question}

\begin{question}
\type{INTEGER}
\chapter{DEFAULT_CHAPTER}
\difficulty{MODERATE}
\text{A hollow sphere is released from rest on a rough inclined plane as shown. If the coefficient of friction is $\mu=0.2$ and the incline is long enough, find the velocity (in m/s) of point of contact after time $t=1 \mathrm{sec}$. (g $=10 \mathrm{~m} / \mathrm{s}^{2}$ )\\ \includegraphics[max width=\textwidth, center]{34fecb1e-3dd5-40ba-8a82-f13b292ab7b9-12(1)}}
\answer{?}
\end{question}

\begin{question}
\type{INTEGER}
\chapter{DEFAULT_CHAPTER}
\difficulty{MODERATE}
\text{A glass hemisphere of refractive index $4 / 3$ and of radius 4 cm is placed on a plane mirror. A point object is placed at distance ' d ' on axis of this sphere as shown. If the final image is at infinity, find the value of ' d ' in cm .\\ \includegraphics[max width=\textwidth, center]{34fecb1e-3dd5-40ba-8a82-f13b292ab7b9-12}}
\answer{?}
\end{question}

\begin{question}
\type{INTEGER}
\chapter{DEFAULT_CHAPTER}
\difficulty{MODERATE}
\text{A uniform rod of length ' $\ell$ ' starts rotating about one end on a rough horizontal table. Its initial angular velocity is $\omega$. The angle $\theta$ through which the rod rotates before coming to rest, is given by $\theta=\frac{\omega^{2} \ell}{n \mu \mathrm{~g}}$. What is the value of n ?}
\answer{?}
\end{question}
